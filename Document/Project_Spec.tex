\documentclass[12pt]{article}

\usepackage{graphicx}
\usepackage{paralist}
\usepackage{amsfonts}
\usepackage{amsmath}
\usepackage{hhline}
\usepackage{booktabs}
\usepackage{multirow}
\usepackage{multicol}
\usepackage{url}
\usepackage{hyperref}
\usepackage{listings}

\oddsidemargin -10mm
\evensidemargin -10mm
\textwidth 160mm
\textheight 200mm
\renewcommand\baselinestretch{1.0}

\pagestyle {plain}
\pagenumbering{arabic}

\newcounter{stepnum}

%% Comments

\usepackage{color}

\newif\ifcomments\commentstrue

\ifcomments
\newcommand{\authornote}[3]{\textcolor{#1}{[#3 ---#2]}}
\newcommand{\todo}[1]{\textcolor{red}{[TODO: #1]}}
\else
\newcommand{\authornote}[3]{}
\newcommand{\todo}[1]{}
\fi

\newcommand{\wss}[1]{\authornote{blue}{SS}{#1}}

\title{Specification}
\author{COMPSCI 2XB3}

\begin {document}

\maketitle
This Module Interface Specification (MIS) document contains three main modules; Queue, RedBlackBST, EarthquakeT,
and CSVreader. It also contains a small EarthquakeBag module.

\newpage

\section* {EarthquakeBag Module}

\subsection*{Template Module}

EarthquakeBag is seq of EarthquakeT

\newpage

\section* {Generic Queue Module}

\subsection*{Generic Template Module inherits Iterable(T)}

Queue(T)

\subsection* {Uses}

None

\subsection* {Syntax}

\subsubsection* {Exported Constants}

None

\subsubsection* {Exported Types}

Queue = ?

\subsubsection* {Internal Types}

Node = ?\\

\noindent \textit{\# Internal Node type has a link to next item in the queue.}

\subsubsection* {Exported Access Programs}

\begin{tabular}{| l | l | l | p{5cm} |}
\hline
\textbf{Routine name} & \textbf{In} & \textbf{Out} & \textbf{Exceptions}\\
\hline
Queue & ~ & Queue & ~\\
\hline
isEmpty & ~ & $\mathbb{B}$ & ~\\
\hline
enqueue & T & ~ & ~\\
\hline
toString & ~ & String & ~\\
\hline
start & ~ & ~ & ~\\
\hline
next & ~ & T & NoSuchElementException\\
\hline
\end{tabular}

\newpage

\subsection* {Semantics}

\subsubsection* {State Variables}

first: Node\\
last: Node\\
$n: \mathbb{N}$\\
\underline{$s$: seq of T}\\

\noindent \textit{\# For simplification, the linked-node structure is represented by seq of T.}\\
\noindent \textit{\# s[1] is the first Node.}\\
\noindent \textit{\# s[$n$] is the last Node.}

\subsubsection* {State Invariant}

None

\subsubsection* {Assumptions}

None

\subsubsection* {Access Routine Semantics}

\noindent Queue():
\begin{itemize}
\item transition: first, last, n := null, null, 0
\item output: $out := \mbox{self}$
\item exception: none
\end{itemize}

\noindent isEmpty():
\begin{itemize}
\item output: $out := (n = 0) \Rightarrow True | True \Rightarrow False$
\item exception: None
\end{itemize}

\noindent enqueue(item):
\begin{itemize}
\item output: $out := s || \mbox{item}$
\item exception: None
\end{itemize}

\noindent toString():
\begin{itemize}
\item output: $out := out || (\forall\, i: \mathbb{N}| i\in[1..n] : s[i])$
\item exception: None
\end{itemize}

\medskip

\noindent \underline{Iterator Methods}:\\

$i: \mathbb{N}$\\

\noindent start():
\begin{itemize}
\item transition: $i := 0$
\item exception: none
\end{itemize}

\noindent next():
\begin{itemize}
\item transition-output: $i, out := i + 1, s[i]$
\item exception: $( i > n) \Rightarrow \text{NoSuchElementException}$
\end{itemize}

\subsubsection* {Considerations}

When an instance of Queue is iterated in a loop, an iterator consisting of these two methods is returned, 
and the start() method is call initially, and for the successive iterations next() method is call.\\

\newpage

\section* {Generic RedBlackBST Module}

\subsection*{Generic Template Module}

RedBlackBST(T with Comparable(T), V)

\subsection* {Uses}

Queue

\subsection* {Syntax}

\subsubsection* {Exported Types}

RedBlackBST = ?

\subsubsection* {Internal Types}

Node = ?

\noindent State Variables of Node:\\
key: Key,
\underline {lst: seq of V},
left: Node,
right: Node,
color: B,
size: N\\

\noindent \textit{\# Internal Node type was modified to store a seq of V.}

\subsubsection* {Exported Access Programs}

\begin{tabular}{| l | l | l | l |}
\hline
\textbf{Routine name} & \textbf{In} & \textbf{Out} & \textbf{Exceptions}\\
\hline
RedBlackBST & ~ & RedBlackBST & \\
\hline
size & ~ & $\mathbb{N}$ & ~\\
\hline
isEmpty & ~ & $\mathbb{B}$ & ~\\
\hline
get & T & seq of V & ~\\
\hline
put & T, V & ~ & ~\\
\hline
min & ~ & T & ~\\
\hline
max & ~ & T & ~\\
\hline
keys & ~ & Queue of T & ~\\
\hline
keys &  T, T & Queue of T & ~\\
\hline
values & T, T & Queue of V & ~\\
\hline
\end{tabular}

\subsection* {Semantics}

\subsubsection* {State Variables}

root: Node\\
RED: $\mathbb{B}$\\
BLACK: $\mathbb{B}$\\
\underline{$s$: set of $\langle T, V \rangle$}\\

\noindent \textit{\# For simplification, the linked-node structure is represented by set of $\langle T, V \rangle$.}\\

\subsubsection* {State Invariant}

RED = True\\
BLACK = False

\subsubsection* {Assumptions}

None

\subsubsection* {Access Routine Semantics}

RedBlackBST():
\begin{itemize}
\item transition: None
\item output: $out := \mathit{self}$
\item exception: None
\end{itemize}

\noindent size():
\begin{itemize}
\item output: $out := root.size$
\item exception: None
\end{itemize}

\noindent isEmpty():
\begin{itemize}
\item output: $out := (root = null) \Rightarrow True | True \Rightarrow False$
\item exception: None
\end{itemize}

\noindent get(key):
\begin{itemize}
\item output: $out$ := L where $\langle x, L \rangle \in s \wedge (x.key = \mathit{key})$
\item exception: None
\end{itemize}

\noindent put(key, val):
\begin{itemize}
\item transition: $s := \{ \langle x, L \rangle : \langle T, V \rangle |
  \langle x, L \rangle \in s : (x.key = \mathit{key} \Rightarrow \langle x, L || [val] \rangle
   | \text{True} \Rightarrow \langle x, L \rangle) \}$

\item exception: None
\end{itemize}

\noindent min():
\begin{itemize}
\item output: $out$ := smallest key in $s$
\item exception: None
\end{itemize}

\noindent max():
\begin{itemize}
\item output: $out$ := largest key in $s$
\item exception: None
\end{itemize}

\noindent keys():
\begin{itemize}
\item output: $out := out || (\forall \langle x, L \rangle : \langle T, V \rangle | \langle x, L \rangle \in s : x.key)$
\item exception: None
\end{itemize}

\noindent keys(lo, hi):
\begin{itemize}
\item output: $out := out || (\forall \langle x, L \rangle : \langle T, V \rangle | \langle x, L \rangle \in s \wedge
lo \le x.key \le hi : x.key)$
\item exception: None
\end{itemize}

\noindent values(lo, hi):
\begin{itemize}
\item output: $out := out || (\forall \langle x, L \rangle : \langle T, V \rangle | \langle x, L \rangle \in s \wedge
lo \le x.key \le hi : L)$
\item exception: None
\end{itemize}

\newpage

\section* {EarthquakeT Module}

\subsection* {Template Module}

EarthquakeT

\subsection* {Uses}

LocalDateTime, PointT

\subsection* {Syntax}

\subsubsection* {Exported Types}

EarthquakeT = ?\\

\noindent ColorRating = \{ NOCOLOR, ZERO, PURPLE, BLUE, GREEN, YELLOW, ORANGE, RED \}\\

\noindent MagType = \{ M5, mb, MB, Mb, MC, Mc, mc, ML, MLSn, MN, MS, MW, Ms, Mw, BLANK \}\\

\noindent \textit{\# EarthquakeT implements Comparable(EarthquakeT)}

\subsubsection* {Exported Constants}

None

\subsubsection* {Exported Access Programs}

\begin{tabular}{| l | l | l | l |}
\hline
\textbf{Routine name} & \textbf{In} & \textbf{Out} & \textbf{Exceptions}\\
\hline
EarthquakeT & String, String, LocalDateTime, & EarthquakeT &\\
\hhline{|~|~|~|}
~ &$\mathbb{R}, \mathbb{R}, \mathbb{R}, \mathbb{R}$, MagType, ColorRating & ~ &\\
\hline
getNameOfProv & ~ & String & \\
\hline
getPlace & ~ & String & \\
\hline
getPointT & ~ & PointT & \\
\hline
getMag & ~ & $\mathbb{R}$ & \\
\hline
getDph & ~ & $\mathbb{R}$ & \\
\hline
getMagitudeType & ~ & MagType & \\
\hline
getDate & ~ & LocalDateTime & \\
\hline
getColor & ~ & ColorRating & \\
\hline
compareTo & EarthquakeT & $\mathbb{Z}$ & \\
\hline
equals & EarthquakeT & $\mathbb{B}$ & \\
\hline
\end{tabular}

\subsection* {Semantics}

\subsubsection* {State Variables}

place: String\\ 
nameOfProv: String\\
date: LocalDateTime\\
lat: $\mathbb{R}$\\
lng: $\mathbb{R}$\\
dph: $\mathbb{R}$\\
mag: $\mathbb{R}$\\
magnitudeType: MagType\\
color: ColorRating

\subsubsection* {State Invariant}

None

\subsubsection* {Assumptions}

Two earthquakes are not the same if they happened to have two different dates or two different places recorded.

\medskip

\subsubsection* {Access Routine Semantics}

EarthquakeT(place, prov, date, lat, lng, dph, mag, mgT, color):
\begin{itemize}
\item transition: \\ lat, lng, place, nameOfProv, date, dph, mag, magnitudeType, color :=\\
  lat, lng, place, prov, date, dph, mag, mgT, color
\item output: $\mathit{out} := \mathit{self}$
\item exception: None
\end{itemize}

\noindent getNameOfProv():
\begin{itemize}
\item output: $out$ := nameOfProv
\item exception: None
\end{itemize}

\noindent getPlace():
\begin{itemize}
\item output: $out$ := place
\item exception: None
\end{itemize}

\noindent getPointT():
\begin{itemize}
\item output: $out$ := PointT(lat, lng)
\item exception: None
\end{itemize}

\noindent getMag():
\begin{itemize}
\item output: $out$ := mag
\item exception: None
\end{itemize}

\noindent getDph():
\begin{itemize}
\item output: $out$ := dph
\item exception: None
\end{itemize}

\noindent getMagitudeType():
\begin{itemize}
\item output: $out$ := magnitudeType
\item exception: None
\end{itemize}

\noindent getDate():
\begin{itemize}
\item output: $out$ := date
\item exception: None
\end{itemize}

\noindent getColor():
\begin{itemize}
\item output: $out$ := color
\item exception: None
\end{itemize}

\noindent compareTo(eq):
\begin{itemize}
\item output: $out$ := an integer value according to the following table.\\

\begin{tabular}{|p{4cm}|r|}
\hhline{~|-|}
\multicolumn{1}{r|}{} & \multicolumn{1}{l|}{$out :=$}\\
\hhline{|-|-|}
$this.\mbox{mag} < \mbox{eq.mag}$ & -1 \\
\hhline{|-|-|}
$this.\mbox{mag} > \mbox{eq.mag}$ & 1 \\
\hhline{|-|-|}
$this.\mbox{mag} = \mbox{eq.mag}$ & 0 \\
\hhline{|-|-|}
\end{tabular}

\item exception: None
\end{itemize}

\noindent equals(that):
\begin{itemize}
\item output: $out := ( \mbox{sameDate} \wedge \mbox{samePoint} \wedge \mbox{samePlace}\\
 \wedge \mbox{sameDepth} \wedge \mbox{sameMagValue} \wedge \mbox{sameMagType} \wedge \\
\mbox{sameEqClass}) \Rightarrow True | True \Rightarrow False$
\item exception: None
\end{itemize}

\newpage

\subsection*{Local Functions}

\noindent sameDate: $\mbox{EarthquakeT} \rightarrow \mathbb{B}$\\
$\mbox{sameDate}(d) \equiv (d.\mbox{date}) = (this.\mbox{date})$\\
\noindent \wss{Returns true if the given EarthquakeT object has the same date as the current.}\\

\noindent samePoint: $\mbox{EarthquakeT} \rightarrow \mathbb{B}$\\
$\mbox{samePoint}(d) \equiv (d.\mbox{Point}) = (this.\mbox{Point})$\\
\noindent \wss{Returns true if the given EarthquakeT object has the same Point as the current.}\\

\noindent samePlace: $\mbox{EarthquakeT} \rightarrow \mathbb{B}$\\
$\mbox{samePlace}(d) \equiv (d.\mbox{place}) = (this.\mbox{place})$\\
\noindent \wss{Returns true if the given EarthquakeT object has the same place as the current.}\\

\noindent sameDepth: $\mbox{EarthquakeT} \rightarrow \mathbb{B}$\\
$\mbox{sameDepth}(d) \equiv |d.\mbox{dph} - this.\mbox{dph}| < 0.0000001$\\
\noindent \wss{returns true if the given EarthquakeT object has the same depth value\\
 as the current within the tolerance.}\\

\noindent sameMagValue: $\mbox{EarthquakeT} \rightarrow \mathbb{B}$\\
$\mbox{sameMagValue}(d) \equiv |d.\mbox{mag} - this.\mbox{mag}| < 0.0000001$\\
\noindent \wss{returns true if the given EarthquakeT object has the same magnitude value\\
 as the current within the tolerance.}\\

\noindent sameMagType: $\mbox{EarthquakeT} \rightarrow \mathbb{B}$\\
$\mbox{sameMagType}(d) \equiv (d.\mbox{magnitudeType}) = (this.\mbox{magnitudeType})$\\
\noindent \wss{Returns true if the given EarthquakeT object has the same magnitude type\\
 as the current.}\\

\noindent sameEqClass: $\mbox{EarthquakeT} \rightarrow \mathbb{B}$\\
$\mbox{sameEqClass}(d) \equiv (d.\mbox{color}) = (this.\mbox{color})$\\
\noindent \wss{Returns true if the given EarthquakeT object has the same class as the current.}\\

\newpage

\section* {CSVreader Module}

\subsection* {Module}

CSVreader

\subsection* {Uses}

CityPostT, CityT, EarthquakeT,\\
EarthquakeT.ColorRating, EarthquakeT.MagType,\\
EarthquakeBag, GeoCollection, RedBlackBST

\subsection* {Syntax}

\subsubsection* {Exported Constants}

None

\subsubsection* {Exported Access Programs}

\begin{tabular}{| l | l | l | l |}
\hline
\textbf{Routine name} & \textbf{In} & \textbf{Out} & \textbf{Exceptions}\\
\hline
readEarthquakes & String, EarthquakeBag & ~ & ~\\
\hline
readEarthquakesBST & String, RedBlackBST & ~ & ~\\
\hline
readPopulation & String, GeoCollection & ~ & ~\\
\hline
readCityPosition & String, seq of CityPostT & ~ & ~\\
\hline
rmFirstLastQuote & String & String & ~\\
\hline
generateColorRating & $\mathbb{R}$ & ColorRating & ~\\
\hline
fullProvName & String & String & ~\\
\hline
\end{tabular}

\subsection* {Semantics}

\subsubsection* {Environment Variables}

None

\subsubsection* {State Variables}

None

\subsubsection* {State Invariant}

None

\subsubsection* {Assumptions}

None

\subsubsection* {Access Routine Semantics}

\noindent readEarthquakes(filename, bag):
\begin{itemize}
\item transition: Read each line of the earthquake csv file and convert to EarthquakeT object, which is stored in a EarthquakeBag.
\item exception: None
\end{itemize}

\noindent readEarthquakesBST(filename, bst):
\begin{itemize}
\item transition: Read each line of the earthquake csv file and convert to EarthquakeT object, which is stored in a RedBlackBST.
\item exception: None
\end{itemize}

\noindent readPopulation(filename, geoCollec):
\begin{itemize}
\item transition: Read each line of the population csv file and convert to CityT object, which is stored in a GeoCollection HashMap.
\item exception: None
\end{itemize}

\noindent readCityPosition(filename, cityPostList):
\begin{itemize}
\item transition: Read each line of the city coordinates csv file and convert to CityPostT object,
 which is stored in a list of cities.
\item exception: None
\end{itemize}

\noindent rmFirstLastQuote(cell):
\begin{itemize}
\item transition: Remove first and last double quotations from a string.
\item exception: None
\end{itemize}

\noindent generateColorRating(cell4):
\begin{itemize}
\item transition: Generate an enum ColorRating type based on the magnitude of earthquake.
\item exception: None
\end{itemize}

\noindent fullProvName(nameP):
\begin{itemize}
\item output: a new province name similar to the following table.

\begin{tabular}{|p{3cm}|p{3.5cm}|l|}
\hhline{~|-|-|}
\multicolumn{1}{l|}{} & nameP $=$ & \multicolumn{1}{l|}{$out :=$}\\
\hhline{|-|-|-|}
$|\mbox{nameP}| = 2$ & ON & Ontario\\
\hhline{|~|-|-|}
 & QC, PQ & Quebec\\
\hhline{|~|-|-|}
 & NS & Nova Scotia\\
\hhline{|~|-|-|}
 & NB & New Brunswick\\
\hhline{|~|-|-|}
 & MB & Manitoba\\
\hhline{|~|-|-|}
 & BC & British Columbia\\
\hhline{|~|-|-|}
 & PE & Prince Edward Island\\
\hhline{|~|-|-|}
 & SK & Saskatchewan\\
\hhline{|~|-|-|}
 & AB & Alberta\\
\hhline{|~|-|-|}
 & NL & Newfoundland and Labrador\\
\hhline{|~|-|-|}
 & NU & Nunavut\\
\hhline{|~|-|-|}
 & NT & Northwest Territories\\
\hhline{|~|-|-|}
 & YT & Yukon\\
\hhline{|~|-|-|}
 & AK & Alaska\\
\hhline{|~|-|-|}
 & WA & Washington\\
\hhline{|~|-|-|}
 & default & UNLOCATED\\
\hhline{|-|-|-|}
$|\mbox{nameP}| \neq 2$ & VANCOUVER ISLAND & British Columbia\\
\hhline{|~|-|-|}
& SOUTHERN QUEBEC & Quebec\\
\hhline{|~|-|-|}
& default & UNLOCATED\\
\hhline{|-|-|-|}
\end{tabular}\\\\

\item exception: None
\end{itemize}

\subsubsection* {Considerations}

There are a number of different variations of geolocation names in the earthquake csv file, for these an appropriate province name
should be assigned. For any that could not be matched to a province name, UNLOCATED should be assigned.\\


\end {document}